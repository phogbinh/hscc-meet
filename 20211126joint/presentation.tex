\documentclass{beamer}
\usepackage{boondox-calo} % lowercase calligraphic letters
\usepackage[backend=biber]{biblatex}

\beamertemplatenavigationsymbolsempty
\usetheme{Warsaw}
\addbibresource{reference.bib}
\setbeamertemplate{bibliography item}{\insertbiblabel}

\newcommand
{\E} % [e]xpected value
{\mathbb{E}}

\newcommand
{\channelState}
{s}

\newcommand
{\indicatorFull}
{^{\pi, \channelState}_{u, m}}

\newcommand
{\discountRatio}
{\frac{\rvUrllcLoadSlot}{\vaResourceSlot}}

\newcommand
{\discountFunc}
{h\left[ \discountRatio \right]}

% [c]on[s]tants
\newcommand
{\csChannelStatePeakCapacity}
{\mathfrak{c}^{\channelState}_{u}}

\newcommand
{\csMinislot}
{\mathfrak{m}}

\newcommand
{\csFB} % [F]requency [B]andwidth
{\mathfrak{f}}

\newcommand
{\csSlotTime}
{\mathfrak{t}}

\newcommand
{\csMinislotTime}
{\mathfrak{t}^{\prime}}

% [va]riables
\newcommand
{\vaResourceMinislot}
{\phi\indicatorFull}

\newcommand
{\vaResourceMF} % [M]inislot [F]requency
{\phi_{f, m}}

\newcommand
{\vaResourceSlot}
{\phi^{\pi, \channelState}_{u}}

\newcommand
{\vaFB} % [F]requency [B]andwidth
{\zeta\indicatorFull}

% [r]andom [v]ariables
\newcommand
{\rvChannelStateCapacity}
{C^{\pi, \channelState}_{u}}

\newcommand
{\rvUrllcLoadMF} % [M]inislot [F]requency
{L^{\pi}_{f, m}}

\newcommand
{\rvUrllcLoadMinislot}
{L\indicatorFull}

\newcommand
{\rvUrllcLoadSlot}
{L^{\pi, \channelState}_{u}}

\title{Report: Joint Scheduling of URLCC and eMBB Traffic in 5G Wireless Networks}
\author{Tran Phong Binh}
\institute{High-Speed Communication and Computing Laboratory, Department of Computer Science, National Tsing Hua University}
\date{\today}

\begin{document}

\begin{frame}
  \titlepage
\end{frame}

\begin{frame}
  \frametitle{Table of Contents}
  \tableofcontents
\end{frame}

\section{Linear Model}
\begin{frame}
  \frametitle{Linear Model}
  This presentation describes the core idea of \cite{linear}.
\end{frame}

\begin{frame}
  \frametitle{System Model}
  Without URLLC superposition/puncturing,
  \begin{equation*}
    \rvChannelStateCapacity = \csChannelStatePeakCapacity \vaResourceSlot
  \end{equation*}
  With URLLC superposition/puncturing,
  \begin{align*}
    \rvChannelStateCapacity & = \csChannelStatePeakCapacity \vaResourceSlot - \csChannelStatePeakCapacity \vaResourceSlot \discountFunc \\
    & = \csChannelStatePeakCapacity \vaResourceSlot \left( 1 - \discountFunc \right)
  \end{align*}
\end{frame}

\begin{frame}
  \frametitle{System Model}
  With linear URLLC superposition/puncturing,
  \begin{equation*}
    \discountFunc = \discountRatio
  \end{equation*}
  That is,
  \begin{align*}
    \rvChannelStateCapacity & = \csChannelStatePeakCapacity \vaResourceSlot \left( 1 - \discountRatio \right)\\
    & = \csChannelStatePeakCapacity \left( \vaResourceSlot - \rvUrllcLoadSlot \right)
  \end{align*}
  Hence,
  \begin{equation}
    \label{eq:model}
    \E \rvChannelStateCapacity = \csChannelStatePeakCapacity \left( \vaResourceSlot - \E \rvUrllcLoadSlot \right)
  \end{equation}
\end{frame}

\begin{frame}
  \frametitle{System Model}
  \begin{itemize}
    \item We assume linear URLLC superposition/puncturing.
    \item We place URLLC loads uniformly over frequency bandwidth across minislots.
    \item I assume $\vaResourceMinislot$ is fixed for a given set of $\left(\channelState, u, m\right)$, for $\E \rvUrllcLoadMinislot$ in equation \ref{eq:exUrllcLoadMinislot} to be a proper constant.
    \item However, I note that such assumption is incorrect because the model aims to modify $\vaResourceMinislot$ over time.
  \end{itemize}
\end{frame}

\begin{frame}
  \frametitle{My Derivation}
  \begin{itemize}
    \item Each minislot has an URLLC demand of $\E D_{m} = \frac{\E D}{\csMinislot}$.
    \item Each frequency $f$ in a minislot has an URLLC load of $\E \rvUrllcLoadMF$.
  \end{itemize}
  \begin{align}
    \label{eq:exUrllcLoadMF}
    \int_{0}^{\csFB} \E \rvUrllcLoadMF df & = \E D_{m}\nonumber\\
    \E \rvUrllcLoadMF \int_{0}^{\csFB} df & = \frac{\E D}{\csMinislot}\nonumber\\
    \E \rvUrllcLoadMF \csFB               & = \frac{\E D}{\csMinislot}\nonumber\\
    \E \rvUrllcLoadMF                 & = \frac{\E D}{\csFB \csMinislot}
  \end{align}
\end{frame}

\begin{frame}
  \frametitle{My Derivation}
  \begin{itemize}
    \item Each frequency $f$ in a minislot has a resource of $\vaResourceMF$.
  \end{itemize}
  \begin{align}
    \label{eq:vaResourceMF}
    \int_{0}^{\csFB} \vaResourceMF df & = \frac{1}{\csMinislot}\nonumber\\
    \vaResourceMF                 & = \frac{1}{\csFB \csMinislot}
  \end{align}
  \begin{itemize}
    \item By equations \ref{eq:exUrllcLoadMF} and \ref{eq:vaResourceMF},
  \end{itemize}
  \begin{align*}
    \frac{\E \rvUrllcLoadMF}{\vaResourceMF} & = \E D\\
    \E \rvUrllcLoadMF                       & = \vaResourceMF \E D
  \end{align*}
\end{frame}

\begin{frame}
  \frametitle{My Derivation}
  On the other hand,
  \begin{align*}
    \csFB \csSlotTime   & = 1\\
    \csFB \csMinislotTime \csMinislot & = 1\\
    \csMinislotTime     & = \frac{1}{\csFB \csMinislot}
  \end{align*}
  Furthermore,
  \begin{align*}
    \vaFB \csMinislotTime & = \vaResourceMinislot\\
    \vaFB   & = \vaResourceMinislot \csFB \csMinislot
  \end{align*}
\end{frame}

\begin{frame}
  \frametitle{My Derivation}
  Thence,
  \begin{align}
    \label{eq:exUrllcLoadMinislot}
    \E \rvUrllcLoadMinislot & = \int_{0}^{\vaFB} \E \rvUrllcLoadMF df\nonumber\\
    & = \int_{0}^{\vaResourceMinislot \csFB \csMinislot} \vaResourceMF \E D df\nonumber\\
    & = \E D \int_{0}^{\vaResourceMinislot \csFB \csMinislot} \vaResourceMF df\nonumber\\
    & = \vaResourceMinislot \E D
  \end{align}
  Similar derivations for slot give
  \begin{equation}
    \label{eq:relation}
    \E \rvUrllcLoadSlot = \vaResourceSlot \E D
  \end{equation}
\end{frame}

\begin{frame}
  \frametitle{Conclusion}
  By equations \ref{eq:model} and \ref{eq:relation},
  \begin{align}
    \label{eq:exChannelStateCapacity}
    \E \rvChannelStateCapacity & = \csChannelStatePeakCapacity \left( \vaResourceSlot - \vaResourceSlot \E D \right)\nonumber\\
    & = \csChannelStatePeakCapacity \vaResourceSlot \left( 1 - \E D \right)
  \end{align}
  Equation \ref{eq:exChannelStateCapacity} shows that URLLC superposition/puncturing linear model has fixed channel capacity; that is, URLLC scheduling policy can be chosen to be uniform random replacement, and we only have to optimize eMBB scheduling policy.
\end{frame}

\begin{frame}
    \frametitle{Reference}
    \printbibliography[heading=none]
\end{frame}

\end{document}
